\documentclass[UTF8]{ctexart}
\usepackage{geometry}
\geometry{margin=1.5cm, vmargin={0pt,1cm}}
\setlength{\topmargin}{-1cm}
\setlength{\paperheight}{29.7cm}
\setlength{\textheight}{25.3cm}

% useful packages.
\usepackage{amsfonts}
\usepackage{amsmath}
\usepackage{amssymb}
\usepackage{amsthm}
\usepackage{enumerate}
\usepackage{graphicx}
\usepackage{multicol}
\usepackage{fancyhdr}
\usepackage{layout}
\usepackage{listings}
\usepackage{float, caption}

\lstset{
    basicstyle=\ttfamily, basewidth=0.5em
}

% some common command
\newcommand{\dif}{\mathrm{d}}
\newcommand{\avg}[1]{\left\langle #1 \right\rangle}
\newcommand{\difFrac}[2]{\frac{\dif #1}{\dif #2}}
\newcommand{\pdfFrac}[2]{\frac{\partial #1}{\partial #2}}
\newcommand{\OFL}{\mathrm{OFL}}
\newcommand{\UFL}{\mathrm{UFL}}
\newcommand{\fl}{\mathrm{fl}}
\newcommand{\op}{\odot}
\newcommand{\Eabs}{E_{\mathrm{abs}}}
\newcommand{\Erel}{E_{\mathrm{rel}}}

\begin{document}

\pagestyle{fancy}
\fancyhead{}
\lhead{吴柏辰, 3230105440}
\chead{数据结构与算法 堆排序算法作业}
\rhead{Dec.2nd, 2024}

\section{测试思路}

自己编写的算法名称为HeapSort。首先创建四个作业要求中的序列,然后分别计时测试。

\section{测试的结果}

对比了测试的时间,结果如下表格所示

\begin{table}[htbp]
\centering
\caption{算法时间比较}
\begin{tabular}{|c|c|c|}
\hline
算 法 & HeapSort & std::sort\_heap \\ \hline
v1用时 & 75.2651ms & 47.9119ms \\ \hline
v2用时 & 39.5669ms & 9.04086ms \\ \hline
v3用时 & 43.5356ms & 43.5817ms \\ \hline
v4用时 & 81.18ms & 46.1392ms \\ \hline
\end{tabular}
\end{table}


\end{document}