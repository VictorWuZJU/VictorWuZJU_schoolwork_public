\documentclass{article}
\usepackage{ctex} % 支持中文
\usepackage{booktabs} % 表格美化
\usepackage{amsmath} % 数学公式
\usepackage{graphicx} % 图片插入
\usepackage{fancyhdr}

\begin{document}

\pagestyle{fancy}
\fancyhead{}
\lhead{吴柏辰, 3230105440}
\chead{LIS算法简述}
\rhead{Dec.23rd, 2024}

\section{综述}

一个整数序列 sequence,我们要在这个序列中找到最长的严格递增子序列。dp[i] 表示以第 i 个元素结尾的最长递增子序列的长度,prev[i] 表示在最长递增子序列中,第 i 个元素的前一个元素的索引。maxLength 用来记录目前找到的最长递增子序列的长度,maxIndex 用来记录这个最长递增子序列的最后一个元素的索引。使用两层循环来填充 dp 和 prev 数组。外层循环遍历序列中的每个元素 i。内层循环比较当前元素 i 和之前的所有元素 j(j < i)。如果 sequence[j] 小于 sequence[i],说明 i 可以接在 j 后面形成递增子序列,我们更新 dp[i] 为 dp[j] + 1(即 j 结尾的最长递增子序列长度加一),并且更新 prev[i] 为 j。如果更新后的 dp[i] 大于 maxLength,说明我们找到了更长的递增子序列,更新 maxLength 和 maxIndex。回溯找到最长递增子序列:从 maxIndex 开始,根据 prev 数组回溯,找到最长递增子序列的所有元素,并将它们添加到 lis 数组中。由于我们是逆序添加元素的,所以最后需要将 lis 数组反转,以得到正确的顺序。函数最长递增子序列的长度和子序列本身。

\section{伪代码}
以下是算法的C++风格的伪代码:

\begin{verbatim}
pair<int, vector<int>> 最长递增子序列(vector<int>& sequence) {
    int n = sequence.size();
    vector<int> dp(n, 1); // 存储每个位置结束的LIS的长度
    vector<int> prev(n, -1); // 存储子序列中的前一个索引
    vector<int> lis; // 存储最长递增子序列
    int maxLength = 0; // 存储LIS的长度
    int maxIndex = 0; // 存储LIS起始位置的索引

    // 动态规划求解LIS
    for (int i = 0; i < n; ++i) {
        for (int j = 0; j < i; ++j) {
            if (sequence[j] < sequence[i] && dp[j] + 1 > dp[i]) {
                dp[i] = dp[j] + 1;
                prev[i] = j;
            }
        }
        // 检查是否需要更新最长递增子序列的长度和起始索引
        if (dp[i] > maxLength) {
            maxLength = dp[i];
            maxIndex = i;
        }
    }

    // 回溯找到最长递增子序列
    int index = maxIndex;
    while (index != -1) {
        lis.push_back(sequence[index]); // 添加元素到结果序列
        index = prev[index]; // 移动到前一个元素
    }
    reverse(lis.begin(), lis.end()); // 反转序列以得到正确的顺序

    return {maxLength, lis}; // 返回长度和子序列本身
}
\end{verbatim}


\section{示例}
考虑序列 \{10, 9, 2, 5, 3, 7, 101, 18\}。该算法将找到LIS \{2, 3, 7, 101\},长度为4。以下是算法的逐步执行过程:

\begin{enumerate}
    \item 初始化dp数组和prev数组:
    \begin{verbatim}
    dp = [1, 1, 1, 1, 1, 1, 1, 1]
    prev = [-1, -1, -1, -1, -1, -1, -1, -1]
    \end{verbatim}
    \item 动态规划填充dp和prev数组:
    \begin{verbatim}
    对于 i = 2 (9), j = 0 (10), dp[2] = 1, prev[2] = -1
    对于 i = 3 (2), j = 0 (10), dp[3] = 1, prev[3] = -1
    对于 i = 4 (5), j = 2 (9), dp[4] = 2, prev[4] = 2
    对于 i = 5 (3), j = 3 (2), dp[5] = 2, prev[5] = 3
    对于 i = 6 (7), j = 4 (5), dp[6] = 3, prev[6] = 4
    对于 i = 7 (101), j = 5 (3), dp[7] = 4, prev[7] = 5
    \end{verbatim}
    \item 根据prev数组回溯最长递增子序列:
    \begin{verbatim}
    index = 7 (101), lis = [101]
    index = 5 (7), lis = [101, 7]
    index = 4 (5), lis = [101, 7, 5]
    index = 3 (2), lis = [101, 7, 5, 2]
    \end{verbatim}
    \item 最终得到的最长递增子序列为 \{2, 5, 7, 101\}。
\end{enumerate}

\section{时间复杂度}

算法中有两个嵌套的循环,外层循环遍历序列中的每个元素,内层循环则对每个元素之前的元素进行比较。内层循环中的每次比较操作都是常数时间的操作,即 O(1)。由于外层循环运行 n-1 次,内层循环的总运行次数是 1 + 2 + 3 + ... + (n-1),这个求和公式的结果是 O(n\^2)。

\end{document}
