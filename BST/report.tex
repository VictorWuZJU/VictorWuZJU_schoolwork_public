\documentclass[UTF8]{ctexart}
\usepackage{geometry, CJKutf8}
\geometry{margin=1.5cm, vmargin={0pt,1cm}}
\setlength{\topmargin}{-1cm}
\setlength{\paperheight}{29.7cm}
\setlength{\textheight}{25.3cm}

% useful packages.
\usepackage{amsfonts}
\usepackage{amsmath}
\usepackage{amssymb}
\usepackage{amsthm}
\usepackage{enumerate}
\usepackage{graphicx}
\usepackage{multicol}
\usepackage{fancyhdr}
\usepackage{layout}
\usepackage{listings}
\usepackage{float, caption}

\lstset{
    basicstyle=\ttfamily, basewidth=0.5em
}

% some common command
\newcommand{\dif}{\mathrm{d}}
\newcommand{\avg}[1]{\left\langle #1 \right\rangle}
\newcommand{\difFrac}[2]{\frac{\dif #1}{\dif #2}}
\newcommand{\pdfFrac}[2]{\frac{\partial #1}{\partial #2}}
\newcommand{\OFL}{\mathrm{OFL}}
\newcommand{\UFL}{\mathrm{UFL}}
\newcommand{\fl}{\mathrm{fl}}
\newcommand{\op}{\odot}
\newcommand{\Eabs}{E_{\mathrm{abs}}}
\newcommand{\Erel}{E_{\mathrm{rel}}}

\begin{document}

\pagestyle{fancy}
\fancyhead{}
\lhead{吴柏辰, 3230105440}
\chead{利用detachMin函数实现remove}
\rhead{Nov.4th, 2024}

\section{测试程序的设计思路}

remove的设计前半段和旧版的remove没什么区别,最大的区别在于增加了一个currentparent节点变量用于管理当前节点的父节点,这是为了将删去最小节点的右子树接在当前(x值所在节点)节点的父节点的左侧或右侧,然后分别考虑四种情况:x节点没有子节点,只有左子树,只有右子树,有左右两个子树。前三种情况是简单的,其操作逻辑也和旧版remove逻辑类似,但是这里也使用了currentparent节点进行了指针修改。detachMin函数的功能是传入节点指针t,然后返回以t为根的树的最小节点,并修改树的结构,如果树的最小节点是根节点,会更新根节点的指针,如果最小节点还有右子树,则直接将最小节点的父节点链接上右子树,这也是通过追踪currentparent指针实现的。

\section{测试的结果}

测试结果一切正常。正确输出了删除后的树,没有改变节点元素之间的相对大小。

\end{document}

%%% Local Variables: 
%%% mode: latex
%%% TeX-master: t
%%% End: 
