\documentclass[UTF8]{ctexart}
\usepackage{geometry, CJKutf8}
\geometry{margin=1.5cm, vmargin={0pt,1cm}}
\setlength{\topmargin}{-1cm}
\setlength{\paperheight}{29.7cm}
\setlength{\textheight}{25.3cm}

% useful packages.
\usepackage{amsfonts}
\usepackage{amsmath}
\usepackage{amssymb}
\usepackage{amsthm}
\usepackage{enumerate}
\usepackage{graphicx}
\usepackage{multicol}
\usepackage{fancyhdr}
\usepackage{layout}
\usepackage{listings}
\usepackage{float, caption}

\lstset{
    basicstyle=\ttfamily, basewidth=0.5em
}

% some common command
\newcommand{\dif}{\mathrm{d}}
\newcommand{\avg}[1]{\left\langle #1 \right\rangle}
\newcommand{\difFrac}[2]{\frac{\dif #1}{\dif #2}}
\newcommand{\pdfFrac}[2]{\frac{\partial #1}{\partial #2}}
\newcommand{\OFL}{\mathrm{OFL}}
\newcommand{\UFL}{\mathrm{UFL}}
\newcommand{\fl}{\mathrm{fl}}
\newcommand{\op}{\odot}
\newcommand{\Eabs}{E_{\mathrm{abs}}}
\newcommand{\Erel}{E_{\mathrm{rel}}}

\begin{document}

\pagestyle{fancy}
\fancyhead{}
\lhead{吴柏辰, 3230105440}
\chead{数据结构与算法第四次作业}
\rhead{Oct.20th, 2024}

\section{测试程序的设计思路}

首先创建链表(在测试代码中节点储存数据类型为int,也可以改为其他类型)命名为intList然后插入三个元素(可更改),用迭代器将其逐个输出,然后测试size和empty函数(观察cout输出就好了),至此完成了动态迭代器的测试,清空后再次添加元素使得链表非空,测试const迭代器,测试了insert和erase操作。考虑到可能出现的位于链表头尾的极端情况,测试哨兵节点功能,在头部和尾部进行了插入和删除操作。

\section{测试的结果}

测试结果一切正常。按照上面设计的顺序,依次完成了链表size,empty函数以及insert和erase函数,这是通过迭代器实现的,而迭代器中的自减操作是作为作业补充编写的,故迭代器(大概)运行正确。

我用 valgrind 进行测试,发现没有发生内存泄露。


\end{document}

%%% Local Variables: 
%%% mode: latex
%%% TeX-master: t
%%% End: 
