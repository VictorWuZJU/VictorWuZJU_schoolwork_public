\documentclass{article}
\usepackage{ctex} % 支持中文
\usepackage{booktabs} % 表格美化
\usepackage{amsmath} % 数学公式
\usepackage{graphicx} % 图片插入
\usepackage{fancyhdr}

\begin{document}

\pagestyle{fancy}
\fancyhead{}
\lhead{吴柏辰, 3230105440}
\chead{数据结构与算法 堆排序算法作业}
\rhead{Dec.22nd, 2024}

\section{引言}

头文件提供了括号匹配检查函数、运算符(Operations)连续使用的检查函数,将中缀表达式转化成后缀表达式的函数,将后缀表达式的值求算的函数

\section{括号匹配检查函数Brackets\_Paring\_Check}

检查传入的字符串表达式中的括号是否正确配对。使用一个栈来跟踪左括号,当遇到右括号时,检查栈顶的左括号是否与之匹配并弹出。如果括号不匹配或有未匹配的括号,函数会打印错误信息并返回false。

\section{符号连续使用检查函数Continous\_Usage\_of\_Brackets}

\subsection{作用}检查表达式中是否有连续的运算符。如果发现连续的运算符,函数会打印错误信息并返回false。

\subsection{注意}这个函数不能处理连续的符号使用,这两个符号之间有空格的情况!例如“ + + ”类型。

\section{valid\_check}

综合检查上述二者。

\section{中缀表达式转后缀表达式Infix\_to\_Postfix}

\subsection{内置数据结构}这个函数内置有一个栈Symbols\_Container存放操作符和

\subsection{运算符优先级priority}加乘除运算设为1,乘除运算设为2。

\subsection{括号或操作符判断函数}在外部声明并定义isBracket(char digit), isRightBracket(char digit), isLeftBracket(char digit)这些函数分别判断一个字符是否是括号、右括号或左括号。isOperator函数判断是否为操作符。

\subsection{算法实现}从左到右扫描中缀表达式的每个字符。如果字符是空格,忽略它。如果字符是数字或小数点,读取整个数字(包括小数点),直到遇到非数字非小数点字符,然后将数字添加到后缀表达式中。如果字符是左括号,将其压入栈中。如果字符是右括号,弹出栈中的运算符并添加到后缀表达式中,直到遇到左括号。左括号被弹出并丢弃。如果字符是运算符,弹出栈中所有优先级大于或等于当前运算符的运算符,并添加到后缀表达式中,然后将当前运算符压入栈。最后,如果栈中还有符号,依次弹出加到末尾。

\section{计算后缀表达式evaluatePostfix}
从左到右扫描后缀表达式的每个元素。如果是数字,将其压入栈中。如果是运算符,从栈中弹出所需数量的操作数(通常是两个),执行运算,然后将结果压回栈中。重复步骤2和3,直到表达式的所有元素都被扫描完毕。栈顶的元素即为后缀表达式的计算结果。

\section{测试函数main.cpp}

测试了上述中的几种错误类型,抛出错误信息并不输出后缀表达式和结果,并给定使用者一个接口输入一个表达式(推荐不要带空格)

\end{document}
